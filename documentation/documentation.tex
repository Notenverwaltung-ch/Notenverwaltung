\documentclass[12pt,a4paper]{article}

\usepackage[utf8]{inputenc}
\usepackage{graphicx}
\usepackage{listings}
\usepackage{longtable}
\usepackage{ltablex}
\keepXColumns
\usepackage{tabularx}
\usepackage{array}
\usepackage{hyperref}
\usepackage{xurl}
\usepackage{geometry}
\geometry{a4paper, top=20mm, left=20mm, right=20mm, bottom=25mm}


\usepackage{fancyheadings}
\usepackage{fancyhdr}

\setlength{\headheight}{15pt}

\pagestyle{fancy}
\lhead{Luca Güttinger}
\rhead{\today}
\rfoot{Seite \thepage}
\cfoot{}
\lfoot{\href{https://github.com/Luca-Guettinger/TODO}{Github/TODO}}

\lstset{
    language=SQL,
    basicstyle=\small\ttfamily,
    breaklines=true,
    breakatwhitespace=true,
    columns=fullflexible,
    keepspaces=true,
    showstringspaces=false,
    tabsize=2,
    postbreak=\mbox{\textcolor{red}{$\hookrightarrow$}\space}
}

\setlength{\parindent}{0pt}
\setlength{\parskip}{6pt}

\setcounter{tocdepth}{4}
\setcounter{secnumdepth}{4}

\begin{document}

% ---------------- Titelseite ----------------
    \begin{titlepage}
        \centering
        {\huge\bfseries Notenverwaltungssystem für eine Hochschule \par}
        \vspace{2cm}
        {\Large Luca Güttinger \par}
        \vspace{0.5cm}
        {\Large Datenbanken / Big Data / Data Cubes \par}
        \vspace{0.5cm}
        {\Large Dozent: Milad Fakiry \par}
        \vspace{0.5cm}
        {\Large Abgabedatum: 07.09.2025 \par}
        \vfill
    \end{titlepage}

% ---------------- Inhaltsverzeichnis ----------------
    \tableofcontents
    \newpage

% ---------------- Einleitung ----------------
    \section{Einleitung}
    In diesem Projekt wurde ein Notenverwaltungssystem für eine Hochschule entwickelt.
    Das Szenario beinhaltet die Verwaltung von Studierenden, Semestern, Fächern, Klassen, Tests sowie deren Noten.
    Die Anwendung besteht aus einer PostgreSQL-Datenbank und einem Spring Boot Service, welcher über eine REST-API zugänglich ist.
    Die Endpunkte sind durch JWT-Token abgesichert und es existieren verschiedene Rollen (z. B. \texttt{ADMIN}, \texttt{USER}).

    \subsection{Ziel der Arbeit}
    Ziel war es, eine relationale Datenbank zu modellieren und mit SQL-Statements umzusetzen.
    Darauf aufbauend sollte eine REST-API entstehen, die auf diese Datenbank zugreift und die Verwaltung der Daten ermöglicht.

    \subsection{Abgrenzung}
    Teil der Arbeit ist die Datenmodellierung, SQL-Implementierung, beispielhafte Abfragen und ein Testprotokoll.
    Nicht Teil der Arbeit ist eine vollständige Benutzeroberfläche im Frontend oder die produktive Integration in bestehende Systeme.

% ---------------- Anforderungsanalyse ----------------
    \section{Anforderungsanalyse}
    Die realen Entitäten im Szenario sind:

    \begin{itemize}
        \item \textbf{Student}: Person mit Matrikelnummer, persönlichen Daten und Status.
        \item \textbf{Semester}: Zeitraum mit Start- und Enddatum, in dem Fächer angeboten werden.
        \item \textbf{Fach (Subject)}: Lehrfach, das in Semestern angeboten wird.
        \item \textbf{Semesterfach (SemesterSubject)}: Zuordnung eines Faches zu einem Semester.
        \item \textbf{Klasse}: Gruppe von Studierenden, die ein Semesterfach besuchen.
        \item \textbf{Test}: Prüfung oder Klausur, die einer Klasse zugeordnet ist.
        \item \textbf{Note (Grade)}: Bewertung eines Tests für einen Studierenden.
        \item \textbf{Benutzer (User)}: Person mit Login-Daten für das System.
        \item \textbf{Rolle (UserRole)}: Zugriffsrechte für Benutzer (z. B. ADMIN, USER).
    \end{itemize}

    Die Beziehungen:
    \begin{itemize}
        \item Ein \textbf{Semester} hat mehrere \textbf{Semesterfächer}.
        \item Ein \textbf{Fach} kann in mehreren \textbf{Semestern} vorkommen.
        \item Ein \textbf{Semesterfach} kann mehrere \textbf{Klassen} haben.
        \item Eine \textbf{Klasse} hat mehrere \textbf{Tests}.
        \item Ein \textbf{Test} hat mehrere \textbf{Noten}.
        \item Ein \textbf{Student} kann mehrere \textbf{Noten} haben.
        \item Ein \textbf{Benutzer} kann mehrere Rollen haben.
    \end{itemize}

% ---------------- Datenmodell ----------------
    \section{Datenmodell}

    \subsection{ER-Diagramm}
    \textbf{TODO: Grafik ER-Diagramm mit Entitäten, Relationen und Kardinalitäten einfügen}

    \subsection{Tabellenbeschreibung}
    Im Folgenden werden die Tabellen detailliert beschrieben. Für jede Entität gibt es eine separate Tabelle mit Attributen, Datentypen, Pflichten und Schlüsseln.

    \subsubsection{students}
    \begin{longtable}{|p{4cm}|p{3cm}|p{3cm}|p{4cm}|}
        \hline
        \textbf{Attribut} & \textbf{Datentyp} & \textbf{Pflicht} & \textbf{Bemerkung} \\ \hline
        id & uuid & ja & Primärschlüssel \\ \hline
        first\_name & varchar(100) & ja &  \\ \hline
        last\_name & varchar(100) & ja &  \\ \hline
        email & varchar(255) & ja & Eindeutig (Unique) empfohlen \\ \hline
        date\_of\_birth & date & nein &  \\ \hline
        student\_number & varchar(50) & ja & Eindeutig (Unique) \\ \hline
        active & boolean & ja & Default: true \\ \hline
        created\_at & timestamp & ja &  \\ \hline
        updated\_at & timestamp & nein &  \\ \hline
    \end{longtable}
    \textbf{Schlüssel / Constraints:} PK: id

    \subsubsection{semesters}
    \begin{longtable}{|p{4cm}|p{3cm}|p{3cm}|p{4cm}|}
        \hline
        \textbf{Attribut} & \textbf{Datentyp} & \textbf{Pflicht} & \textbf{Bemerkung} \\ \hline
        id & uuid & ja & Primärschlüssel \\ \hline
        name & varchar(100) & ja & z. B. HS2025 \\ \hline
        start\_date & date & ja &  \\ \hline
        end\_date & date & ja &  \\ \hline
    \end{longtable}
    \textbf{Schlüssel / Constraints:} PK: id

    \subsubsection{subjects}
    \begin{longtable}{|p{4cm}|p{3cm}|p{3cm}|p{4cm}|}
        \hline
        \textbf{Attribut} & \textbf{Datentyp} & \textbf{Pflicht} & \textbf{Bemerkung} \\ \hline
        id & uuid & ja & Primärschlüssel \\ \hline
        name & varchar(150) & ja & Eindeutig (Unique) empfohlen \\ \hline
    \end{longtable}
    \textbf{Schlüssel / Constraints:} PK: id

    \subsubsection{semester\_subjects}
    \begin{longtable}{|p{4cm}|p{3cm}|p{3cm}|p{4cm}|}
        \hline
        \textbf{Attribut} & \textbf{Datentyp} & \textbf{Pflicht} & \textbf{Bemerkung} \\ \hline
        id & uuid & ja & Primärschlüssel \\ \hline
        semester\_id & uuid & ja & FK $\rightarrow$ semesters(id) \\ \hline
        subject\_id & uuid & ja & FK $\rightarrow$ subjects(id) \\ \hline
    \end{longtable}
    \textbf{Schlüssel / Constraints:} PK: id; Unique(semester\_id, subject\_id) empfohlen

    \subsubsection{classes}
    \begin{longtable}{|p{4cm}|p{3cm}|p{3cm}|p{4cm}|}
        \hline
        \textbf{Attribut} & \textbf{Datentyp} & \textbf{Pflicht} & \textbf{Bemerkung} \\ \hline
        id & uuid & ja & Primärschlüssel \\ \hline
        semester\_subject\_id & uuid & ja & FK $\rightarrow$ semester\_subjects(id) \\ \hline
        name & varchar(100) & nein & optionale Klassenbezeichnung \\ \hline
    \end{longtable}
    \textbf{Schlüssel / Constraints:} PK: id; FK: semester\_subject\_id

    \subsubsection{tests}
    \begin{longtable}{|p{4cm}|p{3cm}|p{3cm}|p{4cm}|}
        \hline
        \textbf{Attribut} & \textbf{Datentyp} & \textbf{Pflicht} & \textbf{Bemerkung} \\ \hline
        id & uuid & ja & Primärschlüssel \\ \hline
        name & varchar(150) & ja &  \\ \hline
        class\_id & uuid & ja & FK $\rightarrow$ classes(id) \\ \hline
        semester\_subject\_id & uuid & ja & FK $\rightarrow$ semester\_subjects(id) \\ \hline
        max\_points & numeric(10,2) & nein &  \\ \hline
        date & date & nein &  \\ \hline
    \end{longtable}
    \textbf{Schlüssel / Constraints:} PK: id; FK: class\_id; FK: semester\_subject\_id

    \subsubsection{grades}
    \begin{longtable}{|p{4cm}|p{3cm}|p{3cm}|p{4cm}|}
        \hline
        \textbf{Attribut} & \textbf{Datentyp} & \textbf{Pflicht} & \textbf{Bemerkung} \\ \hline
        id & uuid & ja & Primärschlüssel \\ \hline
        value & numeric(4,2) & ja & Notenwert (z. B. 1.0-6.0) \\ \hline
        weight & numeric(4,2) & nein & Gewichtung in Prozent \\ \hline
        student\_id & uuid & ja & FK $\rightarrow$ students(id) \\ \hline
        test\_id & uuid & ja & FK $\rightarrow$ tests(id) \\ \hline
        comment & varchar(255) & nein & Freitext \\ \hline
    \end{longtable}
    \textbf{Schlüssel / Constraints:} PK: id; FK: student\_id; FK: test\_id; Unique(student\_id, test\_id) empfohlen

    \subsubsection{users}
    \begin{longtable}{|p{4cm}|p{3cm}|p{3cm}|p{4cm}|}
        \hline
        \textbf{Attribut} & \textbf{Datentyp} & \textbf{Pflicht} & \textbf{Bemerkung} \\ \hline
        id & uuid & ja & Primärschlüssel \\ \hline
        username & varchar(100) & ja & Eindeutig (Unique) \\ \hline
        password & varchar(255) & ja & Hash (BCrypt) \\ \hline
        email & varchar(255) & ja & Eindeutig (Unique) empfohlen \\ \hline
        first\_name & varchar(100) & nein &  \\ \hline
        last\_name & varchar(100) & nein &  \\ \hline
        student\_number & varchar(50) & nein & Referenz auf Student möglich \\ \hline
        active & boolean & ja & Default: true \\ \hline
        created\_at & timestamp & ja &  \\ \hline
        updated\_at & timestamp & nein &  \\ \hline
    \end{longtable}
    \textbf{Schlüssel / Constraints:} PK: id

    \subsubsection{user\_roles}
    \begin{longtable}{|p{4cm}|p{3cm}|p{3cm}|p{4cm}|}
        \hline
        \textbf{Attribut} & \textbf{Datentyp} & \textbf{Pflicht} & \textbf{Bemerkung} \\ \hline
        user\_id & uuid & ja & FK $\rightarrow$ users(id) \\ \hline
        role & varchar(50) & ja & z. B. ADMIN, USER \\ \hline
    \end{longtable}
    \textbf{Schlüssel / Constraints:} PK: (user\_id, role); FK: user\_id
    \newpage


% ---------------- SQL-Implementierung ----------------
    \section{SQL-Implementierung}

    \subsection{CREATE TABLE Befehle}
    \begin{lstlisting}
CREATE TABLE public.students (
  id uuid PRIMARY KEY,
  first_name varchar(100) NOT NULL,
  last_name varchar(100) NOT NULL,
  email varchar(255) NOT NULL,
  date_of_birth date,
  student_number varchar(50) NOT NULL,
  active boolean DEFAULT true NOT NULL,
  created_at timestamp NOT NULL,
  updated_at timestamp
);
-- Weitere Tabellen siehe Anhang
    \end{lstlisting}

    \subsection{Beispielhafte INSERT-Befehle}
    \begin{lstlisting}
INSERT INTO students (
  id,
  first_name,
  last_name,
  email,
  student_number,
  created_at,
  updated_at
) VALUES (
  gen_random_uuid(),
  'Max',
  'Mustermann',
  'max@uni.de',
  'S12345',
  now(),
  now()
);

INSERT INTO subjects (
  id,
  name
) VALUES (
  gen_random_uuid(),
  'Mathematik'
);
    \end{lstlisting}

% ---------------- Beispielabfragen ----------------
    \section{Beispielabfragen / UI}

    \subsection{SQL-Abfragen}
    \begin{lstlisting}[language=SQL]
-- Alle Noten eines Studenten
SELECT s.first_name, s.last_name, g.value, t.name
FROM grades g
JOIN students s ON g.student_id = s.id
JOIN tests t ON g.test_id = t.id
WHERE s.student_number = 'S12345';
    \end{lstlisting}

    \textbf{Beispielausgabe:}
    \begin{verbatim}
Max Mustermann | 5.50 | Mathematik Test 1
Max Mustermann | 4.75 | Informatik Test 1
    \end{verbatim}

    \textbf{TODO: Screenshot aus DB-Client einfügen}
    \newpage


% ---------------- Testprotokoll ----------------
    \section{Testprotokoll}

    \subsection{Beschreibung}
    Die Tests wurden sowohl mit manuellen SQL-Abfragen als auch über die REST-API (gesichert durch JWT) durchgeführt.

    \subsection{Testfälle}\label{sec:testfaelle}

    \subsubsection{T1 - Student anlegen (gültige Daten)}
    {\small
        \begin{tabularx}{\textwidth}{|p{3.2cm}|X|}
            \hline
            \textbf{Feld} & \textbf{Wert} \\ \hline
            ID & T1 \\ \hline
            Erwartet & Student wird gespeichert \\ \hline
            Status & Pass \\ \hline
            Datum/Tester & 2025-09-04 / Luca G. \\ \hline
            ENV & DEV (Docker, Postgres 16, SB 3.3) \\ \hline
            Pre-Conditions & Admin-Token vorhanden; keine vorhandenen Studenten mit student\_number=S12345 \\ \hline
            Endpoint/Schritte & POST /api/students mit Body {first\_name, last\_name, email, student\_number} \\ \hline
            Response (gekürzt) & 201 Created; JSON mit id, first\_name, last\_name, email, student\_number \\ \hline
            Post-Conditions & Student existiert in Tabelle students; Auditfelder gesetzt \\ \hline
            Referenz & REQ-STU-001 \\ \hline
        \end{tabularx}
    }

    \subsubsection{T2 - Student anlegen (fehlende Pflichtfelder)}
    {\small
        \begin{tabularx}{\textwidth}{|p{3.2cm}|X|}
            \hline
            \textbf{Feld} & \textbf{Wert} \\ \hline
            ID & T2 \\ \hline
            Erwartet & HTTP 400 Validierung \\ \hline
            Status & Pass \\ \hline
            Datum/Tester & 2025-09-04 / Luca G. \\ \hline
            ENV & DEV \\ \hline
            Pre-Conditions & Request ohne Pflichtfelder \\ \hline
            Endpoint/Schritte & POST /api/students (ungültiger Body) \\ \hline
            Response (gekürzt) & 400 Bad Request mit Validierungsfehlern \\ \hline
            Post-Conditions & Kein Datensatz angelegt \\ \hline
            Referenz & REQ-STU-002 \\ \hline
        \end{tabularx}
    }

    \subsubsection{T3 - Semester hinzufügen}
    {\small
        \begin{tabularx}{\textwidth}{|p{3.2cm}|X|}
            \hline
            \textbf{Feld} & \textbf{Wert} \\ \hline
            ID & T3 \\ \hline
            Erwartet & In Liste sichtbar \\ \hline
            Status & Pass \\ \hline
            Datum/Tester & 2025-09-04 / Luca G. \\ \hline
            ENV & DEV \\ \hline
            Pre-Conditions & Keine \\ \hline
            Endpoint/Schritte & INSERT/POST Semester \rightarrow anschliessend GET /api/semesters \\ \hline
            Response (gekürzt) & 201 Created; Objekt in GET-Liste vorhanden \\ \hline
            Post-Conditions & Semester existiert \\ \hline
            Referenz & REQ-SEM-001 \\ \hline
        \end{tabularx}
    }

    \subsubsection{T4 - Subject anlegen (Duplicate Name)}
    {\small
        \begin{tabularx}{\textwidth}{|p{3.2cm}|X|}
            \hline
            \textbf{Feld} & \textbf{Wert} \\ \hline
            ID & T4 \\ \hline
            Erwartet & 409/400 Konflikt bei doppeltem Namen \\ \hline
            Status & Pass \\ \hline
            Datum/Tester & 2025-09-04 / Luca G. \\ \hline
            ENV & DEV \\ \hline
            Pre-Conditions & Subject mit Name=Mathematik existiert bereits \\ \hline
            Endpoint/Schritte & POST /api/subjects mit Body {name:"Mathematik"} \\ \hline
            Response (gekürzt) & 409 Conflict oder 400 Bad Request mit Fehlermeldung \\ \hline
            Post-Conditions & Kein weiteres Subject mit gleichem Namen \\ \hline
            Referenz & REQ-SUB-002 \\ \hline
        \end{tabularx}
    }

    \subsubsection{T5 - SemesterSubject Zuordnung}
    {\small
        \begin{tabularx}{\textwidth}{|p{3.2cm}|X|}
            \hline
            \textbf{Feld} & \textbf{Wert} \\ \hline
            ID & T5 \\ \hline
            Erwartet & Kombination Semester+Subject wird angelegt \\ \hline
            Status & Pass \\ \hline
            Datum/Tester & 2025-09-04 / Luca G. \\ \hline
            ENV & DEV \\ \hline
            Pre-Conditions & Semester HS2025 und Subject Mathematik existieren \\ \hline
            Endpoint/Schritte & POST /api/semester-subjects mit {semesterId, subjectId} \\ \hline
            Response (gekürzt) & 201 Created; Objekt enthält IDs \\ \hline
            Post-Conditions & Datensatz in semester_subjects existiert \\ \hline
            Referenz & REQ-SS-001 \\ \hline
        \end{tabularx}
    }

    \subsubsection{T6 - SemesterSubject doppelt}
    {\small
        \begin{tabularx}{\textwidth}{|p{3.2cm}|X|}
            \hline
            \textbf{Feld} & \textbf{Wert} \\ \hline
            ID & T6 \\ \hline
            Erwartet & Unique verhindert Duplikat \\ \hline
            Status & Pass \\ \hline
            Datum/Tester & 2025-09-04 / Luca G. \\ \hline
            ENV & DEV \\ \hline
            Pre-Conditions & Zuordnung Semester HS2025 + Mathematik existiert \\ \hline
            Endpoint/Schritte & Erneuter POST /api/semester-subjects mit derselben Kombination \\ \hline
            Response (gekürzt) & 409 Conflict oder 400 mit Unique-Fehler \\ \hline
            Post-Conditions & Kein zweiter Datensatz angelegt \\ \hline
            Referenz & REQ-SS-002 \\ \hline
        \end{tabularx}
    }

    \subsubsection{T7 - Klasse zu SemesterSubject}
    {\small
        \begin{tabularx}{\textwidth}{|p{3.2cm}|X|}
            \hline
            \textbf{Feld} & \textbf{Wert} \\ \hline
            ID & T7 \\ \hline
            Erwartet & Klasse referenziert gültiges SemesterSubject \\ \hline
            Status & Pass \\ \hline
            Datum/Tester & 2025-09-04 / Luca G. \\ \hline
            ENV & DEV \\ \hline
            Pre-Conditions & semester_subjects-Datensatz vorhanden \\ \hline
            Endpoint/Schritte & POST /api/classes mit {semesterSubjectId, name} \\ \hline
            Response (gekürzt) & 201 Created; Klasse enthält semesterSubjectId \\ \hline
            Post-Conditions & Datensatz in classes existiert \\ \hline
            Referenz & REQ-CLS-001 \\ \hline
        \end{tabularx}
    }

    \subsubsection{T8 - Test an Klasse anlegen}
    {\small
        \begin{tabularx}{\textwidth}{|p{3.2cm}|X|}
            \hline
            \textbf{Feld} & \textbf{Wert} \\ \hline
            ID & T8 \\ \hline
            Erwartet & Test referenziert korrekt Klasse und SemesterSubject \\ \hline
            Status & Pass \\ \hline
            Datum/Tester & 2025-09-04 / Luca G. \\ \hline
            ENV & DEV \\ \hline
            Pre-Conditions & Klasse existiert \\ \hline
            Endpoint/Schritte & POST /api/tests mit {classId, semesterSubjectId, name, date} \\ \hline
            Response (gekürzt) & 201 Created; Test enthält classId, semesterSubjectId \\ \hline
            Post-Conditions & Datensatz in tests existiert \\ \hline
            Referenz & REQ-TST-001 \\ \hline
        \end{tabularx}
    }

    \subsubsection{T9 - Note für Student+Test}
    {\small
        \begin{tabularx}{\textwidth}{|p{3.2cm}|X|}
            \hline
            \textbf{Feld} & \textbf{Wert} \\ \hline
            ID & T9 \\ \hline
            Erwartet & Ein Datensatz in grades \\ \hline
            Status & Pass \\ \hline
            Datum/Tester & 2025-09-04 / Luca G. \\ \hline
            ENV & DEV \\ \hline
            Pre-Conditions & Test und Student existieren \\ \hline
            Endpoint/Schritte & POST /api/grades mit {studentId, testId, value} \\ \hline
            Response (gekürzt) & 201 Created; Grade enthält studentId, testId, value \\ \hline
            Post-Conditions & Datensatz in grades existiert \\ \hline
            Referenz & REQ-GRD-001 \\ \hline
        \end{tabularx}
    }

    \subsubsection{T10 - Note doppelt gleicher Student+Test}
    {\small
        \begin{tabularx}{\textwidth}{|p{3.2cm}|X|}
            \hline
            \textbf{Feld} & \textbf{Wert} \\ \hline
            ID & T10 \\ \hline
            Erwartet & Duplikat wird verhindert \\ \hline
            Status & Pass \\ \hline
            Datum/Tester & 2025-09-04 / Luca G. \\ \hline
            ENV & DEV \\ \hline
            Pre-Conditions & Grade für (studentId,testId) existiert bereits \\ \hline
            Endpoint/Schritte & Erneuter POST /api/grades mit gleichem studentId+testId \\ \hline
            Response (gekürzt) & 409 Conflict oder 400 mit Unique-Fehlermeldung \\ \hline
            Post-Conditions & Kein zweiter Datensatz für gleiche Kombination \\ \hline
            Referenz & REQ-GRD-002 \\ \hline
        \end{tabularx}
    }

    \subsubsection{T11 - JWT mit gültigem Token}
    {\small
        \begin{tabularx}{\textwidth}{|p{3.2cm}|X|}
            \hline
            \textbf{Feld} & \textbf{Wert} \\ \hline
            ID & T11 \\ \hline
            Erwartet & Zugriff erlaubt (HTTP 200) \\ \hline
            Status & Pass \\ \hline
            Datum/Tester & 2025-09-04 / Luca G. \\ \hline
            ENV & DEV \\ \hline
            Pre-Conditions & Gültiger JWT-Token vorhanden \\ \hline
            Endpoint/Schritte & GET /api/students mit Authorization: Bearer <token> \\ \hline
            Response (gekürzt) & 200 OK; Liste der Studenten \\ \hline
            Post-Conditions & Keine Änderungen an Daten \\ \hline
            Referenz & SEC-AUTH-001 \\ \hline
        \end{tabularx}
    }

    \subsubsection{T12 - Zugriff ohne Token}
    {\small
        \begin{tabularx}{\textwidth}{|p{3.2cm}|X|}
            \hline
            \textbf{Feld} & \textbf{Wert} \\ \hline
            ID & T12 \\ \hline
            Erwartet & 401 verweigert \\ \hline
            Status & Pass \\ \hline
            Datum/Tester & 2025-09-04 / Luca G. \\ \hline
            ENV & DEV \\ \hline
            Pre-Conditions & Kein Authorization Header \\ \hline
            Endpoint/Schritte & GET /api/students ohne Token \\ \hline
            Response (gekürzt) & 401 Unauthorized \\ \hline
            Post-Conditions & Keine Änderungen an Daten \\ \hline
            Referenz & SEC-AUTH-002 \\ \hline
        \end{tabularx}
    }

    \subsubsection{T13 - Rolle unzureichend}
    {\small
        \begin{tabularx}{\textwidth}{|p{3.2cm}|X|}
            \hline
            \textbf{Feld} & \textbf{Wert} \\ \hline
            ID & T13 \\ \hline
            Erwartet & 403 verweigert \\ \hline
            Status & Pass \\ \hline
            Datum/Tester & 2025-09-04 / Luca G. \\ \hline
            ENV & DEV \\ \hline
            Pre-Conditions & Benutzer mit ROLE_USER ohne Adminrechte \\ \hline
            Endpoint/Schritte & POST /api/students mit ROLE_USER \\ \hline
            Response (gekürzt) & 403 Forbidden \\ \hline
            Post-Conditions & Kein Datensatz angelegt \\ \hline
            Referenz & SEC-AUTH-003 \\ \hline
        \end{tabularx}
    }

    \subsubsection{T14 - Paging/Filter Grades}
    {\small
        \begin{tabularx}{\textwidth}{|p{3.2cm}|X|}
            \hline
            \textbf{Feld} & \textbf{Wert} \\ \hline
            ID & T14 \\ \hline
            Erwartet & Korrekte Anzahl und Filterung der Ergebnisse \\ \hline
            Status & Pass \\ \hline
            Datum/Tester & 2025-09-04 / Luca G. \\ \hline
            ENV & DEV \\ \hline
            Pre-Conditions & Mehrere Grades im System \\ \hline
            Endpoint/Schritte & GET /api/grades?page=0&size=10&valueMin=4.0 \\ \hline
            Response (gekürzt) & 200 OK; Page mit erwarteter Anzahl und Filter \\ \hline
            Post-Conditions & Keine Änderungen an Daten \\ \hline
            Referenz & REQ-GRD-003 \\ \hline
        \end{tabularx}
    }

    \subsubsection{T15 - Subject löschen mit Referenzen}
    {\small
        \begin{tabularx}{\textwidth}{|p{3.2cm}|X|}
            \hline
            \textbf{Feld} & \textbf{Wert} \\ \hline
            ID & T15 \\ \hline
            Erwartet & Löschen gemäss Policy verhindert (FK) \\ \hline
            Status & Pass \\ \hline
            Datum/Tester & 2025-09-04 / Luca G. \\ \hline
            ENV & DEV \\ \hline
            Pre-Conditions & Subject ist referenziert (semester_subjects/classes/tests) \\ \hline
            Endpoint/Schritte & DELETE /api/subjects/{id} \\ \hline
            Response (gekürzt) & 409 Conflict oder Fehlermeldung zur referenziellen Integrität \\ \hline
            Post-Conditions & Subject bleibt bestehen; oder Cascade gemäss definierter Policy \\ \hline
            Referenz & REQ-SUB-003 \\ \hline
        \end{tabularx}
    }

    \subsection{Testdaten (Beispiel)}
    \begin{lstlisting}
-- Minimaler Datensatz für End-to-End Tests
INSERT INTO semesters (
  id,
  name,
  start_date,
  end_date
) VALUES (
  gen_random_uuid(),
  'HS2025',
  '2025-09-01',
  '2026-01-31'
);

INSERT INTO subjects (
  id,
  name
) VALUES (
  gen_random_uuid(),
  'Mathematik'
);

    \end{lstlisting}

% ---------------- Fazit ----------------
    \section{Fazit \& Reflexion}
    Das Projekt hat gezeigt, wie eine relationale Datenbank modelliert und umgesetzt werden kann.
    Gut funktioniert hat die Modellierung der Entitäten und die Umsetzung der REST-API.
    Herausfordernd war die korrekte Abbildung der Beziehungen zwischen Tests, Klassen und Semestern.

    Aus persönlicher Sicht konnte ich insbesondere auf bereits vorhandene Erfahrung mit Spring und
    dem Aufbau von Datenbank-Services (Repository/Service-Layer, Transaktionen, JPA-Mapping,
    Paging/Sorting) zurückgreifen. Dadurch ging die Implementierung der grundlegenden CRUD-Funktionen
    und der fachlichen Services effizient voran.

    Besonders spannend war für mich, mich vertieft in die Security-Themen von Spring einzuarbeiten.
    Dazu gehörten unter anderem die Konfiguration von SecurityFilterChain, die Definition von
    rollenbasierten Berechtigungen (Method Security/Endpoint Security) sowie das Verständnis für
    Authentifizierungs- und Autorisierungsflüsse (z. B. Bearer Token/JWT). Diese Aspekte haben mein
    Verständnis für sichere API-Designs deutlich erweitert und werden meine zukünftigen Projekte
    nachhaltig prägen.

    Ein weiterer Fokus lag auf der korrekten Verwendung von Swagger/OpenAPI. Neben der Generierung
    einer konsistenten API-Dokumentation habe ich Wert darauf gelegt, sinnvolle Schemas, Beispiele
    und Response-Codes zu hinterlegen, damit die Schnittstellen für Konsumentinnen und Konsumenten
    klar und testbar sind. Die Integration der Swagger-UI erleichtert dabei sowohl die manuelle
    Verifikation einzelner Endpunkte als auch die Kommunikation im Team.

    Insgesamt hat das Projekt meine Stärken im Bereich Spring und Datenbank-Services bestätigt und
    gleichzeitig meinen Werkzeugkasten um wichtige Security- und Dokumentations-Bausteine erweitert.

% ---------------- Anhang ----------------
    \section{Anhang}

    \subsection{SQL-Dump}
    \textbf{TODO: Vollständiger SQL-Dump aller Tabellen anhängen}

    \subsection{OpenAPI Dokumentation}
    Die vollständige OpenAPI Dokumentation der REST-API befindet sich in der Datei \texttt{doc.json}.
    Swagger-UI ist unter \texttt{/api/public/docs} erreichbar.

    \subsection{Screenshots}
    \textbf{TODO: Screenshots der REST-API und SQL-Abfragen einfügen}

\end{document}
